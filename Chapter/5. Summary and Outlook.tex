\section{Summary and Outlook} \label{sec:summary-outlook}
The aim of this thesis was to develop and test a suitable method for the quantitative analysis of deviations caused by temporal discretization of signals as they appear in time-resolved investigations.

In order to place this task in the correlated scientific setting, and after a brief motivation in \cref{sec:introduction}, the theoretical backgrounds of related fields were introduced in \cref{sec:theoretical-foundations}.

In \cref{sec:implementation} the self-developed method for quantitative analysis of deviations due to temporal discretization of signals was presented and its application to three different types of signals (sine wave, square wave and Bat-Signal) was discussed. The temporal discretization and deviation evaluation method can be broken down into two main steps: The first step deals with the actual discretization of the input signal by splitting it into intervals of equal length, the so-called gate length $\tau$, and equal spacing, the so called sampling resolution $t_0$, and averaging over them, whereas the second step evaluates the deviation between the time-discrete and input signal at every data point by means of a cubic spline. Setting the gate length to $\tau = 0.1T$ and the sampling resolution to $t_0 = 0.05T$ results in an averaged and normalized deviation $\overline{d}_{pp}$ that varies anywhere between $\overline{d}_{pp} < 1\%$ for the sine wave signal and $\overline{d}_{pp} \approx 5\%$ for the Bat-Signal. Furthermore, inspecting the Bat-Signal in the frequency domain $f$ instead of the time domain $t$ shows that the self-developed method delivers a more intuitive and quantified approach for complicated signals compared to common signal-theoretical methods for the determination of discretization based deviations.

For experimental comparison, discretized phase slopes of time-resolved electron waves, which were obtained via time-resolved electron holography by interference gating, were examined in \cref{sec:application}. The measured phase slopes for the sine wave signal showed little to no deviation to the time-discrete data points of the oscilloscope signal. In comparison, the square wave signal featured only small deviations between the measured phase slopes and the time-discrete data points during the polarity change of the input signal, which can be attributed to parasitic capacitances and described, in combination with the self-developed method, using a modeling approach, drawing a connection to real physical effects. In detail, the exponential curve initially failed to approximate the oscilloscope signal using the measured capacitance $C = \SI{152 \pm 20}{\pico\farad}$ of the MEMS chip, whereas assuming the capacitance at $C = \SI{50}{\pico\farad}$ resulted in a higher agreement between them, showing that the measured capacitance is more applicable to the whole MEMS chip rather than the single capacitor investigated. At last, the Bat-Signal showed a similar behavior of small deviations between the phase slopes and the time-discrete data points, except for a few asymmetric deviations due to stray fields.

In particular, comparison with experimental data has shown that the self-developed method is a promising approach for quantitative analysis of temporal discretization based deviations. Moreover, quantitative calculations show that the interference gating method along with a gate length of $\tau = 0.1T$ and a sampling resolution of $t_0 = 0.05T$, which is easily within experimental feasibility, is able to sample simplified signals, such as a sine or square wave, with little to no deviation.

The presented method therefore allows for a smart choice of gate length $\tau$ and sampling resolution $t_0$ for future investigations and thus contributes to increasing the efficiency of time-resolved electron-holographic measurements and enabling significant time savings in the measurement process. Furthermore, a deeper investigation of the possibility to describe deviations by means of modeling, in connection with the self-developed method (as in the case of the (dis-)charging capacitor), seems particularly interesting and opens the door for quantitative investigation of dynamic physical processes.