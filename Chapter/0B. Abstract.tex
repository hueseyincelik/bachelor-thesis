\section*{Abstract}
Time-resolved electron holography, with its separate access to amplitude and phase information of reconstructed electron waves, allows for spatially-resolved sampling of dynamic processes in the nanosecond range. The thereby resulting deviation errors, which are dependent on the temporal resolution and the sampling rate, represent a lower bound error of such time-resolved measurements.

The following thesis presents a self-developed, computer-based method for quantifying these deviation errors. In order to verify its validity, the results obtained by applying the method to three different signal types are compared to experimental measurements. The self-developed method allows for optimizations of the measurement parameters, enabling significant time savings in the measurement process, as well as conclusions on physical properties of sophisticated specimens by means of physical (feedback) modeling.

More precisely, this thesis shows that the deviations arising from the temporal discretization strongly depend on the signal type. For sinusoidal signals, measurement parameters that are within experimental feasibility result in deviations below 1\%, while the same choice of measurement parameters for a complicated Bat-Signal results in deviations of around 5\%. Furthermore, a comparison with an exponential modeling approach for a square wave signal applied to a capacitor shows that the experimentally measured capacitance of the specimen is more applicable to the entire circuit rather than the investigated coplanar capacitor, whose capacitance has to be corrected downwards in the exponential model.
\begin{otherlanguage}{ngerman}
\section*{Deutsche Zusammenfassung}
Die zeitaufgelöste Elektronenholographie, mit ihrem getrennten Zugang zur Amplitude und Phase rekonstruierter Elektronenwellen, ermöglicht die ortsaufgelöste Abtastung dynamischer Prozesse im Nanosekundenbereich. Die aus solch einer Abtastung resultierenden Abweichungsfehler, welche abhängig von der Zeitauflösung und der Abtastrate sind, stellen eine untere Fehlergrenze solcher zeitaufgelösten Messungen dar.

Die folgende Arbeit stellt eine selbst entwickelte, computerbasierte Methode zur Quantifizierung dieser Abweichungsfehler vor. Zu ihrer Verifizierung werden die durch sie gewonnen Ergebnisse für drei verschiedene Signaltypen mit experimentellen Messungen verglichen. Die entwickelte Methode erlaubt eine Optimierung der Messparameter, was zu einer realen Zeitersparnis während des Messprozesses führt, sowie Rückschlüsse auf physikalische Eigenschaften komplexer Proben mittels physikalischer (Feedback-)Modellierung ermöglicht.

Genauer kann in dieser Arbeit gezeigt werden, dass die aus der zeitlichen Diskretisierung entstehenden Abweichungen stark vom Signaltyp abhängen. Für sinusartige Signale ergeben sich bei experimentell sinnvollen Messparametern Abweichungen unter 1\%, während die Abweichungen bei einem komplizierten Bat-Signal für die gleiche Wahl von Messparametern bei ca. 5\% liegen. Ferner kann ein Vergleich mit einem exponentiellen Modellierungsansatz für ein an einem Kondensator angelegten Rechtecksignal zeigen, dass die experimentell gemessene Kapazität der Probe eher auf den gesamten Schaltkreis zutrifft als auf den untersuchten koplanaren Kondensator, dessen Kapazität im exponentiellen Modell nach unten hin korrigiert werden muss.
\end{otherlanguage}