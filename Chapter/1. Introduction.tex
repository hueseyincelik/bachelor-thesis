\section{Introduction} \label{sec:introduction}
With an ever-growing interest in the investigation of dynamic processes at the nanoscale, time-resolved transmission electron microscopy (TEM) is becoming increasingly important. Current stroboscopic approaches yield time resolutions from the nanosecond \cite{Bostanjoglo2000,Doemer2003} down to the femtosecond range \cite{Hassan2017,Zewail2010} while still trying to preserve atomic spatial resolution \cite{Kisielowski2008}. These approaches, however, are not compatible with all modern microscopic techniques used in a TEM. Especially for electron holography, with its separate access to amplitude and phase information of reconstructed electron waves \cite{Lehmann2002,Lichte2007}, making it an excellent candidate for measurements of potential distribution in nanoelectronic devices, these approaches turned out to be very disadvantageous in terms of partial coherence \cite{Feist2017}.

A simple, yet promising novel method, called \emph{interference gating}, avoids these problems and allows for robust time-resolved measurements in electron holography in the nanosecond range by switching the interference pattern on and off \cite{Niermann2017,Wagner2019}. This approach allows for a flexible adjustment of the time window over which the signal is measured, the so called \emph{gate length} $\tau$, and the temporal distance between measurements, the so called \emph{sampling resolution} $t_0$, with minimal additional equipment needed \cite{Niermann2017,Wagner2019}, allowing physical processes to be easily sampled.

Unfortunately, sampling a signal (or process) often introduces a deviation between the original and time-discrete signal, which is dependent on the chosen parameters \cite{Gray1998,Stickler1967,Kobayashi2000}. This thesis introduces a flexible and self-developed approach to quantize this, regardless of experimental setup always present, temporal discretization based deviation in time-resolved electron holography for different kinds of signals and parameters $\tau$ and $t_0$.

In detail, \cref{sec:theoretical-foundations} introduces the theoretical foundations behind electron holography, interference gating and time-discrete signals. \Cref{sec:implementation} first presents the self-developed temporal discretization and deviation evaluation method in order to subsequently discuss its results in an application to a sine wave, square wave and Bat-Signal. \Cref{sec:application} deals with the experimental setup and the time-resolved electron holographic measurements of different input signals applied to an object and discusses the deviations between the aforementioned discretization deviations and measured phase slopes for those signals. At last, \cref{sec:summary-outlook} gives a short summary of the results of this thesis and an outlook on further research in this area.